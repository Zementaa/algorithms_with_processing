\documentclass[../mciAusarbeitung.tex]{subfiles}

\usepackage[utf8]{inputenc}
\usepackage[T1]{fontenc}
\usepackage{lmodern}
\usepackage[german]{babel}
\usepackage[fixlanguage]{babelbib}
\selectbiblanguage{german}

\title{Fachpraktikum MCI (01513) - WS 2021/22}
\author{Gruppe 2\\
	Catherine Camier}
\date{\today}

\begin{document}

Pünktlich zum Semesterstart startete für uns auch das Fachpraktikum "`Kooperative algorithmische Kunst"'. Das erste Treffen war vor allem dazu gedacht, alle Gruppenmitglieder (im weiteren Verlauf GM) auf einen Stand zu bringen, die Aufgabe zu erfassen und organisatorische Fragen zu klären.
		
\paragraph{Organisatorisches}
Wir haben uns jeden Montag ab 19 Uhr per Zoom verabredet. Das Treffen lief immer so ab, dass zunächst die Ergebnisse der Woche präsentiert wurden. Im Anschluss wurden Fragen und Probleme geklärt, die sich währenddessen ergeben haben. Und zuletzt wurden neue Aufgaben für die kommende Woche verteilt. Dabei dauerten die Treffen zwischen zwei und vier Stunden, wobei vor allem in der Anfangsphase letzteres der Fall war.\\
Wir nutzten von Beginn an sehr intensiv unser Git-Repository. Hier galt es die IDEs und benutzten Bibliotheken sowie Frameworks auf denselben Stand zu bringen. Durch das Einbinden von Maven gelang dies gut. Unser Repository gliedert sich zunächst einmal in die Bereich dev und doc. Darunter haben wir für jedes Gruppenmitglied einen eigenen Ordner angelegt, in dem gearbeitet werden kann. Im doc-Ordner haben wir uns entschieden, dass wir die Texte, die individuell durch ein GM erstellt werden, per "`subfile"' in das Haupt-tex-Dokument "`mciAusarbeitung.tex"' einbinden. So bleibt das Hauptdokument übersichtlich und die einzelnen GM kommen sich nicht in die Quere. Der dev-Ordner wird in api, distribution und die GM-Ordner unterteilt. Auch hier gilt dann, jeder kann in seinem eigenen Bereich arbeiten.\\
Zusätzlich zur Aufgabenver- und -unterteilung, haben wir uns zeitliche Meilensteine gesetzt, die erreicht werden sollten:
\begin{itemize}
	\item Anfang November: Jedes GM hat drei Algorithmen ausgewählt und implementiert.
	\item Anfang Dezember: Fertigstellung des Konfigurators (Architektur und Design).
	\item Anfang Januar: Fertigstellung der Konfiguration der Algorithmen (und deren Zusammenspiel).
	\item Anfang Februar: Fertigstellung der Dokumentation.
\end{itemize}
Wir konnten alle Meilensteine erreichen, was aber sicher nicht nur der guten Organisation, sondern auch dem Fleiß der GM zu verdanken war.
	
\paragraph{Erläuterung zur Aufteilung der Gruppenarbeit}
Keiner bzw. keine von uns hatte bereits im Vorfeld Erfahrungen mit Processing gesammelt. Was wir in die Gruppe mitbringen konnten, waren Erfahrungen mit Java, Git, Maven und Grundlegendes zu Algorithmen. Wobei sich auch hier die Kenntnisse stark unterschieden. Da wir es für eine gute Idee hielten, dass jeder seinen Kenntnissen nach Arbeiten übernehmen sollte, teilten wir die Gruppenarbeit grob auf in Architektur, Algorithmen und Design. Was jedoch klar war, war dass jedes GM sich mit allen Algorithmen zumindest grundlegend befassen musste, um das Gelingen der Kooperation sicherzustellen. Auch in das Arbeiten mit LaTex mussten wir uns zunächst einarbeiten. Diejenigen Teile der schriftlichen Ausarbeitung, die nicht in die GM-Ordner untergliedert wurden, haben wir in der Gruppe aufgeteilt. Die Autoren sind den einzelnen Kapiteln zu entnehmen.
	
\paragraph{Gesamtergebnis}
Am Anfang des Praktikums waren wir zunächst bezüglich der Komplexität der Aufgabenstellung in Sorge, ob und inwiefern wir im Team genügend Kompetenzen erwerben können, um diese zu meistern. Wir hatten viele Fragen, die wir gemeinsam ausdiskutierten, bevor wir praktisch starten konnten. Die Aufgabenstellung lässt auch viel Spielraum, was die Umsetzung betrifft. Nachdem wir die Komplexität erfasst hatten und die Aufgabe in Teilaufgaben untergliedert hatten, konnten wir dann mit der Umsetzung beginnen. Nach der Einarbeitung in die Algorithmen und in Processing stellten sich schnell erste Erfolge ein. Unsere Gruppendynamik war äußerst positiv, sodass einzelne Misserfolge (welche sich bei der Programmierung nie umgehen lassen und meist sogar zu einem besseren Ergebnis führen) mithilfe der anderen GM aufgefangen werden konnten.\\
Zum Ergebnis der Gruppenleistung lässt sich vor der Bewertung kein abschließendes Fazit ziehen. Für uns alle war das Projekt jedoch ein großer Erfolg. Wir konnten sehr viel dazu lernen und hatten Spaß am Parametrisieren unserer Algorithmen. Es konnten auch wirklich schöne Bilder entstehen. Wir sind uns bewusst, dass wir unsere Algorithmen vor allem bezüglich des Laufzeitverhalten optimieren könnten, dass weitere Kooperationsverhalten implementiert werden könnten und dass mithilfe einer besseren Grafikkarte sicher noch bessere Ergebnisse hätten erzielt werden können. Nichtsdestotrotz sind wir sehr zufrieden mit unserem Ergebnis und hoffen, dass wir den Anforderungen an diesen Kurs gerecht werden konnten.
	
\paragraph{Ausblick}
Bei uns allen entstand der Eindruck, dass die Aufgabenstellung zu offen formuliert war, sodass wir uns eine Art Kick-off-Meeting gewünscht hätten, das gemeinsam mit dem MCI-Team stattgefunden hätte. Auch der Umfang des Praktikums überstieg in der Realität die angesetzten Wochenstunden, was aber vielleicht auch unseren teilweise fehlenden Vorerfahrungen verschuldet sein kann. Processing ist eine mächtige Programmiersprache, die im Bereich Grafik, Simulation und Animation tolle Ergebnisse erzielt. In Verbindung mit den Algorithmen, die zu implementieren waren, konnten Erfahrungen gesammelt werden, die wir in zukünftigen Programmierprojekten anwenden können. Obwohl der Kurs anspruchsvoll und umfangreich war, können wir uns durchaus vorstellen, weitere Projekte in diesem Bereich durchzuführen.

\end{document}